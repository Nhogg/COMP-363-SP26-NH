\documentclass{article}

\usepackage{fancyhdr}
\usepackage{extramarks}
\usepackage{amsmath}
\usepackage{amsthm}
\usepackage{amsfonts}
\usepackage{tikz}
\usepackage[plain]{algorithm}
\usepackage{algpseudocode}

\usetikzlibrary{automata,positioning}

%
% Basic Document Settings
%

\topmargin=-0.45in
\evensidemargin=0in
\oddsidemargin=0in
\textwidth=6.5in
\textheight=9.0in
\headsep=0.25in

\linespread{1.1}

\pagestyle{fancy}
\lhead{\hmwkAuthorName}
\chead{\hmwkClass\ (\hmwkClassInstructor\ \hmwkClassTime): \hmwkTitle}
\rhead{\firstxmark}
\lfoot{\lastxmark}
\cfoot{\thepage}

\renewcommand\headrulewidth{0.4pt}
\renewcommand\footrulewidth{0.4pt}

\setlength\parindent{0pt}

%
% Create Problem Sections
%

\newcommand{\enterProblemHeader}[1]{
    \nobreak\extramarks{}{Problem \arabic{#1} continued on next page\ldots}\nobreak{}
    \nobreak\extramarks{Problem \arabic{#1} (continued)}{Problem \arabic{#1} continued on next page\ldots}\nobreak{}
}

\newcommand{\exitProblemHeader}[1]{
    \nobreak\extramarks{Problem \arabic{#1} (continued)}{Problem \arabic{#1} continued on next page\ldots}\nobreak{}
    \stepcounter{#1}
    \nobreak\extramarks{Problem \arabic{#1}}{}\nobreak{}
}

\setcounter{secnumdepth}{0}
\newcounter{partCounter}
\newcounter{homeworkProblemCounter}
\setcounter{homeworkProblemCounter}{1}
\nobreak\extramarks{Problem \arabic{homeworkProblemCounter}}{}\nobreak{}

%
% Homework Problem Environment
%
% This environment takes an optional argument. When given, it will adjust the
% problem counter. This is useful for when the problems given for your
% assignment aren't sequential. See the last 3 problems of this template for an
% example.
%
\newenvironment{homeworkProblem}[1][-1]{
    \ifnum#1>0
        \setcounter{homeworkProblemCounter}{#1}
    \fi
    \section{Problem \arabic{homeworkProblemCounter}}
    \setcounter{partCounter}{1}
    \enterProblemHeader{homeworkProblemCounter}
}{
    \exitProblemHeader{homeworkProblemCounter}
}

%
% Homework Details
%   - Title
%   - Due date
%   - Class
%   - Section/Time
%   - Instructor
%   - Author
%

\newcommand{\hmwkTitle}{Homework\ \#NUM}
\newcommand{\hmwkDueDate}{DUE DATE}
\newcommand{\hmwkClass}{COMP 363 - Algorithms}
\newcommand{\hmwkClassTime}{}
\newcommand{\hmwkClassInstructor}{Leo Irakliotis}
\newcommand{\hmwkAuthorName}{\textbf{Nathan Hogg}}

%
% Title Page
%

\title{
    \vspace{2in}
    \textmd{\textbf{\hmwkClass:\ \hmwkTitle}}\\
    \normalsize\vspace{0.1in}\small{Due\ on\ \hmwkDueDate}\\
    \vspace{0.1in}\large{\textit{\hmwkClassInstructor\ \hmwkClassTime}}
    \vspace{3in}
}

\author{\hmwkAuthorName}
\date{}

\renewcommand{\part}[1]{\textbf{\large Part \Alph{partCounter}}\stepcounter{partCounter}\\}

%
% Various Helper Commands
%

% Useful for algorithms
\newcommand{\alg}[1]{\textsc{\bfseries \footnotesize #1}}

% For derivatives
\newcommand{\deriv}[1]{\frac{\mathrm{d}}{\mathrm{d}x} (#1)}

% For partial derivatives
\newcommand{\pderiv}[2]{\frac{\partial}{\partial #1} (#2)}

% Integral dx
\newcommand{\dx}{\mathrm{d}x}

% Alias for the Solution section header
\newcommand{\solution}{\textbf{\large Solution}}

% Probability commands: Expectation, Variance, Covariance, Bias
\newcommand{\E}{\mathrm{E}}
\newcommand{\Var}{\mathrm{Var}}
\newcommand{\Cov}{\mathrm{Cov}}
\newcommand{\Bias}{\mathrm{Bias}}

\begin{document}

\maketitle

\pagebreak
\begin{homeworkProblem}
  \subsection*{Background}
  The goal of this assignment is to prove the time complexity of the mergesort algorithm.
  We are given the base complexity formula as:
  \begin{align}
      T(n) = 2T(n / 2) + f(n)
  \end{align}
In this case, $T(n)$ is the time it takes to mergesort an array of size $n$, and $f(n)$ is the time it takes to
assemble two partial solutions of size $n / 2$ to a sorted array with $n$ elements. Generally, $f(n) \approx n$.

For an array with $n = 8$, we can write:
\begin{align}
T(8) = 2T(4) + f(8)
\end{align}
Then, $T(4) = 2T(2)+f(4)$; and $T(2) = 2T(1) + f(2)$; and finally $T(1) = f(1)$. Using the last finding,
we can solve backwards:
\begin{align}
  T(2) &= 2T(1) + f(2) \\
       &= 2f(1) + f(2) \\[1em]
  T(4) &= 2T(2) + f(4) \\
       &= 2(2f(1) + f(2)) + f(4) \\
       &= 4f(1) + 2f(2) + f(4) \\[1em]
  T(8) &= 2T(4) + f(8) \\
       &= 2(4f(1) + 2f(2) + f(4)) + f(8) \\
       &= 8f(1) + 4f(2) + 2f(4) + f(8)
\end{align}
Given that $f(n) \equiv n$, the last equation can be rewritten as:
\begin{align}
T(8) \equiv 8 \times 1 + 4 \times 2 + 2 \times 4 + 8 \\
= 32 = 8 \times 3 \\
= 8(1 + log_{2}8)
\end{align}

We have shown that, for mergesort, $T(n) \equiv nlog_{2}n$ (or, $\mathcal{O} (nlog_{2}n$),

\subsection*{Proof}
We begin by writing:
\begin{align*}
T(n) &= 2T\!\left(\frac{n}{2}\right) + f(n) \\[4pt]
     &= 2\left[ 2T\!\left(\frac{n}{4}\right) + f\!\left(\frac{n}{2}\right) \right] + f(n) \\[4pt]
     &= 2\left[ 2\left[ 2T\!\left(\frac{n}{8}\right) + f\!\left(\frac{n}{4}\right) \right] + f\!\left(\frac{n}{2}\right) \right] + f(n) \\[4pt]
     &= 2\left[ 2\left[ 2\left[ 2T\!\left(\frac{n}{16}\right) + f\!\left(\frac{n}{8}\right) \right] + f\!\left(\frac{n}{4}\right) \right] + f\!\left(\frac{n}{2}\right) \right] + f(n)
\end{align*}




\end{homeworkProblem}

\pagebreak


\end{document}
